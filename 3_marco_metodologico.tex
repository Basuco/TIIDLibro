% Marco Metodologico.
\chapter{Marco Metodológico} \label{chap:Marco Metodologico}

\vspace{5 mm}

		Este capítulo describe la metodología de desarrollo utilizada durante el proyecto de pasantía. Para este proyecto se empleó la metodología ágil SCRUM, debido a que el departamento de producto, el equipo de desarrollo de software, utiliza esta metodología y es con la que han tenido mejores resultados.\cite{SCRUM} 
\newline
\newline
\indent En este proyecto de pasantía, el \textit{Product Owner} fue un consultor de producto y el \textit{Scrum Master} el coordinador de producto, ambos supervisados por la direcotra de productos. A continuación se presentan los componentes que forman parte de esta metodología:
		
\section{Roles} \label{sect:Roles}
 Existen tres roles principales dentro del equipo de trabajo, cada rol posee un conjunto de actividades bien definidas que deben ser realizadas en un período de tiempo específico para que el proyecto pueda culminar de manera exitosa. Estos son los roles:
 
\subsection{Scrum Master} \label{sect:Scrum Master}
El Scrum Master, es responsable de asegurar que el proceso SCRUM sea entendido y aceptado por todos los miembros del equipo, esto se hace haciendo uso de las reglas, prácticas y teorías. Por otro lado, el Scrum Master actúa como mediador entre agentes externos y el equipo de desarrollo, de esta manera se minimiza los problemas que puedan existir con la interacción entre los mismos.\cite{SCRUM}
  
\subsection{Product Owner} \label{sect:Product Owner}
El Product Owner es responsable de maximizar el valor del producto en creación y el trabajo del equipo desarrollador o \textit{development team}, también tiene la visión del producto y trabaja en colaboración con usuarios, clientes y \textit{stakeholders}. El Product Owner es el único responsable de determinar cuales son las funcionalidades a crear en el \textit{product backlog} o pila del producto.\cite{SCRUM}

\subsection{Development Team} \label{sect:Development Team}
Es el equipo de trabajo que se encarga de llevar a cabo el diseño y desarrollo del software requerido. El equipo tiene la responsabilidad de cumplir con las historias del \textit{Product Backlog} para ir desarrollando paso a paso el producto, la manera en como se desarrollan estas historias es una decision tomada por el equipo. Por lo general consta de 3 a 9 personas.\cite{SCRUM}


\section{Eventos} \label{sect:Eventos}
Los eventos en la metodología Scrum son todas las reuniones utilizadas para crear regularidad y actualizar al equipo constantemente del estado del proyecto y las siguientes acciones que se deberian llevar a cabo. Todos estos eventos tienen un tiempo de duración y son aprovechados para revisar y adaptar cada una de las historias desarrolladas. Estos eventos son los siguientes:\cite{SCRUM}

\subsection{Sprint} \label{sect:Sprint}
El \textit{Sprint} es el corazón del \textit{Scrum}, es un período de tiempo donde se desarrollan ciertas historias propuestas en el \textit{Sprint Backlog} y se entrega un producto funcional. Es recomendado que la duración de los sprints sea constante y definida por el equipo de desarrollo en base a su experecia. Un nuevo \textit{Sprint} comienza inmediatamente despues de concluir uno.

\subsection{Sprint Planning} \label{sect:Sprint Planning}
\textit{Sprint Planning}, es la reunión que se realiza al inicio de cada \textit{Sprint} en la que se determinará que historias serán realizadas. El Product Owner, como se mencionó anteriormente, es el encargado de definir las historias y organizarlas según su prioridad, seguidamente el equipo de desarrollo determina las historias que pueden desarrollar.\cite{SCRUM}

\subsection{Daily Sprint Meeting} \label{sect:Daily Sprint Meeting}
Reunión de máximo 15 minutos diaria que sirve para sincronizar y actualizar el plan de actividades para las siguientes 24 horas de desarrollo. Esta reunión se realiza todos los dias y cada miembro resume lo que hizo el día anterior, como le va en el día y que problemas ha presentado.\cite{SCRUM}

\subsection{Sprint Review} \label{sect:Sprint Review}
Se lleva acabo al final de cada \textit{Sprint}, donde se muestran los avances realizados en el producto. Dichos avances son presentados al \textit{Product Owner} y todos los demás interesados (pueden estar presentes clientes finales). 

\subsection{Sprint Retrospective} \label{sect:Sprint Retrospective}
Esta reunión sirve para reflexionar sobre el último Sprint realizado y sirve para inspeccionar los problemas que se tuvieron y buscar una manera de solucionarlos en los sprints venideros. Tiene un maximo de duración de 3 horas.\cite{SCRUM}

\section{Artefactos} \label{sect:Artefactos}
Los artefactos de SCRUM representan trabajo o valor de diversas formas que son útiles para brindar transparencia para la inspección y adaptación. Estos artefactos están diseñados para maximizar el entendimiento de la información.\cite{SCRUM}

\subsection{Product Backlog} \label{sect:Product Backlog}
Product backlog o pila del producto es una lista ordenada de todas las historias que pueden ser necesarias para el producto, con descripciones detalladas y sus prioridades de todos los requisitos funcionales y no funcionales. Esta lista es dinámica y pública para todos los involucrados en el proyecto. La pila del producto es creada por el \textit{Product Owner} durante el \textit{Sprint Planning}. \cite{SCRUM}

\subsection{Sprint Backlog} \label{sect:Sprint Backlog}
Es un documento detallado por parte del equipo de desarrollo, donde
se predicen las actividades necesarias a desarrollar en el siguiente
\textit{sprint} para dar como terminadas ciertas taréas.\cite{SCRUM}