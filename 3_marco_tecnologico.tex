% Marco Teorico.
\chapter{Marco Metodológico} \label{chap:Marco Metodológico}

\vspace{5 mm}

		El propósito de este capítulo es presentar y describir la metodología de desarrollo utilizada durante el proyecto de pasantía, describiendo las actividades y entregables realizados en cada fase del proceso de desarrollo. Para el desarrollo de este proyecto se utilizó el marco de trabajo MSF, siglas que significan “Microsoft Solutions Framework”, sobre el cual se define la metodología de trabajo para ejecución de proyectos, de uso estándar en Triops Solutions.

\section{Metodología utilizada} \label{sect:Metodología utilizada}
		Microsoft Solutions Framework (MSF) es un enfoque personalizable para entregar con éxito soluciones tecnológicas de manera más rápida, con menos recursos humanos y menos riesgos, pero con resultados de más calidad. MSF ayuda a los equipos a enfrentarse directamente a las causas más habituales de fracaso de los proyectos tecnológicos y mejorar así las tasas de éxito, la calidad de las soluciones y el impacto comercial. En la figura 3.1 se observa el esquema de la metodología MSF.

\section{Implementación de la metodología en el proyecto} \label{sect:Implementación de la metodología en el proyecto}
		La metodología de desarrollo utilizada se denomina Microsoft Solutions Framework (MSF) y consta de cinco fases:

\begin{itemize}[noitemsep,nolistsep]
\item Visión: El objetivo fundamental de esta fase es desarrollar un entendimiento claro sobre lo que se necesita dentro del contexto de las restricciones del proyecto. Es por ello, que fueron necesarias numerosas reuniones con miembros de la empresa involucrados en el proyecto. De igual forma, se trabajó en la familiarización con las herramientas de trabajo, así como la búsqueda de conceptos relacionados al negocio.
\item Planeación: Esta fase se enfoca en hacer evolucionar la solución conceptual hasta llegar a diseños y planes tangibles. Esto significó la elaboración de diseños conceptuales del sistema, así como la arquitectura a implementar.
\item Compilar: Esta fase busca compilar los aspectos de la solución de acuerdo con las entregas de la pista de planeación, como diseños, planes, programaciones y requisitos. Esta fase está dividida en tres iteraciones: Desarrollo del módulo de conexión a software de autenticación, Desarrollo del módulo de conexión a interfaz del sistema hotelero y su integración con módulo de conexión, y, Desarrollo de la interfaz web en Django.
\item Estabilización: Esta fase tiene como principal objetivo mejorar la calidad de la solución para satisfacer los criterios de lanzamiento para la implementación en producción. Por ello, se desarrolló un plan de pruebas unitarias, y más tarde de integración que garantizara la funcionalidad del sistema.
\item Despliegue: El objetivo principal de esta fase es integrar una solución correctamente en producción dentro de los entornos designados. Aunque esta fase no estaba contemplada para este proyecto de pasantía se llevó a cabo un procedimiento dedicado a instalar el software en producción. Esto requirió la realización de scripts de instalación que facilitaran la portabilidad del sistema, y su traslado de un ambiente de laboratorio a un ambiente de producción.
\end{itemize}

