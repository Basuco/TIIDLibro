\chapter{Módulo de la Lógica del Negocio}\label{chapter:Módulo de la Lógica del Negocio}

		En este capítulo se presenta la descripción del módulo especializado en la lógica del negocio, software intermediario entre la interfaz IFC8 y el sistema de autenticación FreeRadius. En la primera sección se explica el procedimiento de clasificación de la información obtenida mediante la interfaz. En la segunda sección se describe la ‘conversión’ de dicha información una vez clasificada, a datos exactos que el sistema de autenticación entenderá (en este caso, el apellido de los huéspedes). En la tercera sección se explica el proceso de extensión del acceso a la red una vez que el huésped ha efectuado su salida del hotel. Finalmente, en la cuarta sección se detalla la clasificación de usuarios en los distintos grupos, según sus beneficios.

\section{Gestión de los datos obtenidos} \label{sect:Gestión de los datos obtenidos}
		Debido a que los datos ya clasificados sirven de base para el motor de autenticación FreeRadius, es necesaria su depuración para que sean datos adecuados. Específicamente, se habla del cambio de los apellidos que contengan caracteres especiales como ‘ñ’, vocales acentuadas, e incluso caracteres no pertenecientes al idioma español (ej.: ‘ç’). Esto, además de aumentar las posibilidades de distribución internacional, permite facilidad de acceso a la red a los usuarios finales. 
		
\section{Manejo de la hora de salida de los huéspedes} \label{sect:Manejo de la hora de salida de los huéspedes}
		Si en la clasificación (descrita en la sección 5.1), se determina que el mensaje recibido es de tipo ‘salida’ (es decir, el huésped ha hecho check-out), se debe aplicar una extensión en la hora establecida. Esto se debe a la política del hotel de no suspender el acceso a la red, inmediatamente después que se efectúa la salida. Al usuario se le otorga una extensión de tiempo determinada y configurable por los administradores de NetPass. Luego de finalizada esa extensión, el usuario es eliminado de la base de datos del sistema, por lo que se le restringirá su acceso a la red.
		
\section{Categorización de usuarios} \label{sect:Categorización de usuarios}
		En esta etapa del proceso surge la necesidad de categorizar a los usuarios que han ingresado al hotel (check-in). Dicha clasificación diferencia entre los recursos concedidos a cada usuario dentro de la red (por ejemplo, el uso de sesiones simultáneas, control de ancho de banda). Estas distinciones son otorgadas al motor de autenticación que se encarga de permitir o negar estos beneficios. Existen X tipos de usuarios, los cuales se detallan a continuación.
		
\subsection{Usuario - Plan básico}
\subsection{Usuario - Plan premium}
		
\section{Resultados obtenidos} \label{sect:Resultados obtenidos}
		Como resultado de esta etapa del desarrollo se obtuvo el módulo que gestiona la lógica del negocio. En este punto, el sistema es capaz de amoldar los datos obtenidos desde OPERA PMS (a través del middleware IFC8) y transformarlos según las especificaciones deseadas. Esto, con el fin de despacharlos al motor de autenticación, más adelante.