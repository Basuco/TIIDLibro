\chapter{Conclusiones y Recomendaciones} 

\label{chap:conclusiones}

En este proyecto de pasantía se desarrolló el Asistente para ensamblar aplicaciones móviles usando tecnología web que permite crear aplicaciones móviles multiplataforma usando el \textit{framework} Kiraso de Synergy-GB. El proceso completo incluyó el diseño y desarrollo del \textit{frontend} de la aplicación y su respectiva integración con el \textit{backend} también diseñado y desarrollado durante este proceso. 


Este Asistente para ensamblar aplicaciones permite crear aplicaciones móviles nuevas y estructurarlas gráficamente, configurar los distintos componentes, agregar, consultar y eliminar componentes, ver y descargar la estructura de archivos, editar los archivos de la aplicación, ejecutar la aplicación y ver su vista previa, ver los mensajes de los procesos a tiempo de ejecución, agregar recursos para personalizar las aplicaciones y una gestión básica de usuarios. 


Esta solución permite a los usuarios o desarrolladores reducir los tiempos de desarrollo de aplicaciones móviles, además de ser una forma más amigable de familiarizarse con el \textit{framework} Kiraso. Se espera que, una vez que este Asistente pase a producción dentro de la empresa, se convierta en una herramienta importante que le permita a Synergy-GB disminuir el esfuerzo requerido para desarrollar sus proyectos, aumentando su productividad.


Para el desarrollo del prototipo se utilizaron diversas herramientas entre las que destaca el \textit{stack} de MEAN.js, es decir, MongoDB, Express, AngularJS y NodeJS. Así como los patrones de diseño: inyección de dependencias y observador. El aprendizaje de lo antes mencionado durante el proyecto de pasantía, resultó en una experiencia enriquecedora que permitió aumentar las destrezas del desarrollador. 


Al finalizar el periodo establecido para el desarrollo de la pasantía se obtuvo un prototipo que cubre todos los objetivos planteados tanto por el cliente como por el plan de trabajo, así como también ciertas funcionalidades adicionales de interés para la empresa. 


Este Asistente, que es una extensión del \textit{framework} Kiraso, es una herramienta que ofrece ventajas tanto a Synergy-GB como a sus clientes, además de proveer una forma de trabajo que busca que los desarrolladores hagan productos de calidad y fáciles de mantener, por esto se recomienda altamente seguir desarrollando y trabajando en esta herramienta para que sus mejoras ofrezcan grandes beneficios a la empresa.

Dado que el desarrollo del Asistente para ensamblar aplicaciones móviles no ha sido finalizado en su totalidad, otra recomendación fundamental es la mejora de la gestión de usuarios. Es importante agregar características como seguridad y manejo de roles. Si el \textit{framework} se va a comercializar junto con su Asistente es importante dar distintos niveles de usuarios, con distintos permisos para las acciones a realizar con el Asistente. Además de agregar un administrador que tenga control sobre los distintos usuarios registrados.

También se recomienda a la empresa que el desarrollo del \textit{framework} vaya de la mano siempre con el desarrollo de este Asistente. Es decir, que a medida que se expanda y se agreguen funcionalidades o mejoras al \textit{framework} Kiraso, se realice el desarrollo necesario, tanto en el \textit{frontend} como en el \textit{backend} del Asistente, para que de esta manera en todo momento abarque completamente las funcionalidades del \textit{framework} y se le pueda sacar todo el provecho posible.


Por último, se recomienda que se realicen pruebas exhaustivas que garanticen la calidad del desarrollo.


Esta experiencia de pasantía le permitió al pasante buscar una solución a un problema real en donde tuvo que poner en práctica todos los conocimientos adquiridos a lo largo de la carrera universitaria. Por otro lado, se generó un crecimiento en cuanto a la gestión del tiempo y de cambios en los requerimientos, y estimación de tiempo de las actividades. Además, se dio la oportunidad de aprender nuevas herramientas y tecnologías, así como la oportunidad de trabajar en conjunto con el equipo de desarrollo de la empresa y participar en la toma de decisiones sobre cambios en el \textit{framework} y agregar nuevas funcionalidades al mismo. Todo lo mencionado anteriormente, resultó muy beneficioso para el pasante ya que aumenta sus destrezas y habilidades en el ámbito laboral.

  