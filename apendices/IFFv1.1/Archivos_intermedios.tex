\chapter{Archivos intermedios}

%---------------------------------------------------------------------------------------%
\section {Representación de Dominio}

\label{archivos_intermedios:dom}
El archivo contiene los valores posibles para la variable que se codificó, el formato es 
valor y punto. En el caso de los \textit{flotantes} y \textit{doubles} el valor va entre comillas dobles.
Un ejemplo del nombre del archivo para un nodo con \texttt{id 5} es: \texttt{X5.dom}.
Un ejemplo del contenido del archivo suponiendo que es de tipo entero y tiene valores 
posibles \texttt{10}, \texttt{15} y \texttt{20}, es como el de la figura \ref{fig:ej_dominio}.
\begin{figure}[h]
\begin{lstlisting}[mathescape]
10.
15.
20.
\end{lstlisting}
\caption[Dominio Enteros]
{Dominio Enteros}
\label{fig:ej_dominio}
\end{figure}

%---------------------------------------------------------------------------------------%
\section {Representación de Rango}

\label{archivos_intermedios:siran}
El archivo que representa el rango para una variable independiente en el problema, tiene
en cada linea un valor correspondiente a la variable que se codificó. El nombre del 
archivo por ejemplo para el nodo con el \texttt{id 10} es: \texttt{s\_i\_10.ran}. El contenido
del archivo para una para una variable tipo \emph{string} que tenga valores posibles \texttt{hola},
\texttt{hello} y \texttt{hallo}, luce como la figura \ref{fig:ej_si}.

\begin{figure}[h]
\begin{lstlisting}[mathescape]
hola
hello
hallo
\end{lstlisting}
\caption[Rango Independiente String]
{Rango Independiente String}
\label{fig:ej_si}
\end{figure}

\label{archivos_intermedios:ran}
El archivo que representa el rango para un sistema, tiene los \texttt{id} de los nodos
que están involucrados, todos estos se encuentran al inicio del archivo separados por saltos de linea. Luego el símbolo \texttt{\#} y los
valores que satisfacen el sistema separados por espacio, el orden en el que se encuentran
corresponde al \texttt{id} de las variables que se encuentran al principio del archivo. 
El nombre del archivo por ejemplo para el sistema \texttt{0} es: \texttt{s\_0.ran}. El contenido del 
archivo para un sistema con una variable tipo float con \texttt{id 5} y otra entera con 
\texttt{id 9} y vectores solución \texttt{15.0 10} y \texttt{12.0 10}, luce como la figura \ref{fig:ej_s}.

\begin{figure}[h]
\begin{lstlisting}[mathescape]
5
9
#
15.0 10
12.0 10
\end{lstlisting}
\caption[Rango de Sistema]
{Rango de Sistema}
\label{fig:ej_s}
\end{figure}

%---------------------------------------------------------------------------------------%
\section {Sistemas en \emph{Prolog}}
\label{archivos_intermedios:pl}

Los archivos de \emph{Prolog} generados para resolver los sistemas lucen como el de la figura 
\ref{fig:ej_pl}.

\begin{figure}[h]
\begin{lstlisting}[mathescape]
probar(Vector):-
 nth0(0,Vector,X_4),
 nth0(1,Vector,X_8),
 !,
 (X_4 > 11),
 (X_4 < X_8).

main:- 
 consult(include/prolog/funciones_csp),
 inicializarEnteros('temp/X4.dom', X4),
 inicializarEnteros('temp/X8.dom', X8),
 solucionar([X4,X8],Vectores),
 escribirVectores(Vectores, [4,8], 'temp/s_0.ran').
\end{lstlisting}
\caption[Sistema para \emph{Prolog}]
{Sistema para \emph{Prolog}}
\label{fig:ej_pl}
\end{figure}

En la sección del main lo que hace es: primero cargar un conjunto de funciones auxiliares,
luego cargar los valores que estén en los archivos de dominio \ref{archivos_intermedios:dom}.
Al ya tener todos estos valores cargados, se generan todas las tuplas posibles, a estos se les
llamará ``vectores solución'', si cumplen las restricciones se imprimen estos vectores en el
formato que se describi'o en \ref{archivos_intermedios:ran}.

Lo que hace es que para el predicado probar, se toma el vector que se quiere verificar que cumpla 
la restricción y luego tomar de esa lista de valores que esta en 
\texttt{Vector} y comprobar que \texttt{X\_4} y \texttt{X\_8} cumplan la restricción que se codificó.

